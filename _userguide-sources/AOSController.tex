\setNextFileName{AOSController.html}
\begin{section}{AOS Controller}
\label{sec:aoscontroller}

A primary design goal for the adaptive optimization system is to enable research in online feedback-directed optimization. Therefore, we require the controller implementation to be flexible and extensible. As we gained experience with the system, the controller component went through several major redesigns to better support our goals.

The controller is a single Java thread that runs an infinite event loop. After initializing AOS, the controller enters the event loop and attempts to dequeue an event. If no event is available, the dequeue operation blocks (suspending the controller thread) until an event is available. All controller events implement an interface with a single method: process. Thus, after successfully dequeuing an event the controller thread simply invokes its process method and then, the work for that event having been completed, returns to the top of the event loop and attempts to dequeue another event. This design makes it easy to add new kinds of events to the system (and thus, extend the controller's behavior), as all of the logic to process an event is defined by the event's process method, not in the code of the controller thread.

A further level of abstraction is accomplished by representing the recompilation strategy as an abstract class with several subclasses. The process method of a hot method event invokes methods of the recompilation strategy to determine whether or not a method should be recompiled, and if so at what optimization level. The cost-benefit model itself is also reified in a class hierarchy of models to enable extension and variation. This set of abstractions enable a single controller implementation to execute a variety of strategies.

Another useful mechanism for experimentation is the ability to easily change the input parameters to AOS that define the expected compilation rates and execution speed of compiled code for the various compilers. By varying these parameters, one can easily cause the default multi-level cost-benefit model to simulate a single-level model (by defining all but one optimization level to be unprofitable). One can also explore other aspects of the system, for example the sensitivity of the model to the accuracy of these parameters. We found this capability to be so useful that the system supports a command line argument (\spverb+-X:aos:dna=<filename>+) that causes it to optionally read these parameters from a file.

\end{section}
