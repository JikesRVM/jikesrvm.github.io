\setNextFileName{ExceptionManagement.html}
\begin{section}{Exception Management}
\label{sec:exceptionmanagement}

The runtime has to deal with the relatively small number of hardware signals which can be generated during Java execution. On operating systems other than AIX, an attempt to dereference a \spverb+null+ value (an access to a null value manifests as a read to a small negative address outside the mapped virtual memory address space) will generate a a segmentation fault. This means that the Jikes RVM does not need to generate explicit tests guarding against dereferencing null values and this results in faster code generationg for non-excepting code.

The Jikes RVM handles the signal and reenters Java so that a suitable Java exception handler can be identified, the stack can be unwound (if necessary) and the handler entered in order to deal with the exception. Failing location of a handler, the associated Java thread must be cleanly terminated.

The Jikes RVM actually employs software traps to generate hardware exceptions in a small number of other cases, for example to trap array bounds exceptions. Once again a software only solution would be feasible. However, since a mechanism is already in place to catch hardware exceptions and restore control to a suitable Java handler the use of software traps is relatively simple to support.

Use of a hardware handler enables the register state at the point of exception to be saved by the hardware exception catching routine. If a Java handler is registered in the call frame which generated the exception this register state can be restored before reentry, avoiding the need for the compiler to save register state around potentially excepting instructions. Register state for handlers in frames below the exception frame is automatically saved by the compiler before making a call and so can always be restored to the state at the point of call by the exception delivery code.

The bootloader registers signal handlers which catch \spverb+SEGV+ and \spverb+TRAP+ signals. These handlers save the current register state on the stack, create a special handler frame above the saved register state and return into this handler frame executing \texttt{Run\-ti\-me\-En\-try\-points.de\-li\-ver\-Hard\-wa\-re\-Ex\-cep\-tion()}. This method searches the stack from the excepting frame (or from the last Java frame if the exception occurs inside native code) looking for a suitable handler and unwinding frames which do not contain one. At each unwind the saved register state is reset to the state associated with the next frame. When a handler is found the delivery code installs the saved register state and returns into the handler frame at the start of the handler block.

The Jikes RVM employs some of the same code used by the hardware exception handler to implement the language primitive \spverb+throw()+. This primitive requires a handler to be located and the stack to be unwound so that the handler can be entered. A \spverb+throw+ operation is always translated into a call to \texttt{Run\-ti\-me\-En\-try\-points.a\-throw()} so the unwind can never happens in the handler frame. Hence the register state at the point of re-entry is always saved by the call mechanism and there is no need to generate a hardware exception.

\end{section}
