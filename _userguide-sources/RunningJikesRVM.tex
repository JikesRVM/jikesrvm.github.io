\setNextFileName{RunningJikesRVM.html}
\begin{section}{Running Jikes RVM}
\label{sec:runningjikesrvm}

Jikes\textsuperscript{TM} RVM executes Java virtual machine byte code instructions from \spverb+.class+ files. It does \textit{not} compile Java\textsuperscript{TM} source code. Therefore, you must compile all Java source files into bytecode using a Java compiler.

For example, to run class foo with source code in file foo.java:

\begin{lstlisting}
% javac foo.java
% rvm foo
\end{lstlisting}

The general syntax is

\begin{lstlisting}
rvm [rvm options...] class [args...]
\end{lstlisting}

You may choose from a myriad of options for the rvm command-line. Options fall into two categories: \textit{standard} and \textit{non-standard}. Non-standard options are preceded by "-X:" and differ between virtual machines (e.g. Jikes RVM's options may not be available in HotSpot and vice-versa).

\begin{subsection}{Standard Command-Line Options}

We currently support a subset of the JDK 1.5 standard options. Below is a list of all options and their descriptions. Unless otherwise noted each option is supported in Jikes RVM.

\begin{table}[h]
\centering
\begin{tabular}{p{0.5\textwidth}p{0.4\textwidth}}
Option & Description \\
\spverb+-cp+ or \spverb+-classpath+ (directories and \newline zip/jar files separated by ":") & set search path for application classes and resources \\
\spverb+-D<name>=<value>+ & set a system property \\
-verbose:[class\textbar gc\textbar jni] & enable verbose output \\
-version & print current VM version and terminate the run \\
-showversion & print current VM version and continue running \\
-fullversion & like "-version", but with more information \\
-? or -help & print help message \\
-X & print help on non-standard options \\
-jar & execute a jar file \\
\spverb+-javaagent:<jarpath>[=<options>]+ & load Java programming language agent, see \spverb+java.lang.instrument+ \\
\end{tabular}
\end{table}

\end{subsection}

% TODO continue porting this section

\end{section}